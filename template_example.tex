%========================================================================================%
%					NOTE
%========================================================================================%
% Probably this theme will not compile with standard pdftolatex compiler due to lack of some 
% fonts used in the theme. It is recommended to compile with XeLaTex instead.

% !TEX program = xelatex

%========================================================================================%
%					DOCUMENT SETUP
%========================================================================================%
\documentclass{beamer} 
\usetheme[progressbar=frametitle]{metropolis}

\usepackage{textpos}


%For making diagram and drawing
\usepackage{tikz}
\usetikzlibrary{shapes,arrows, fit, positioning}

%For appendix
\usepackage{appendixnumberbeamer}


\usepackage[export]{adjustbox}

%Some additional graphcs tools
%\usepackage{graphicx}
%\usepackage{datatool}
%\usepackage{animate}

% For someone using the pgfplot tools
%\usepackage{pgfplots}
%\usepgfplotslibrary{dateplot, groupplots}
%\pgfplotsset{compat=1.14}
%\usepgfplotslibrary{fillbetween}

%Remove words break, wrap instead
\usepackage[none]{hyphenat}


%for custom date
\usepackage[english]{babel}
\usepackage[nodayofweek,level]{datetime}

\usepackage{xspace}
\newcommand{\themename}{\textbf{\textsc{metropolis}}\xspace}


%===================================================================================%
%				FRONT PAGE
%===================================================================================%
\titlegraphic{%
\hfill%
\includegraphics[height=1.5cm, valign=c]{logos/ccfd3.png}%
\hspace{15pt}%
\includegraphics[height=1.5cm, valign=c]{logos/symbol-PL.pdf}%
}

\author{\underline{Main author}, Author 2, Author 3} 

\title{Very important title}

\institute{\textbf{Warsaw University of Technology, Faculty of Power and Aeronautical Engineering}}

\date{\vspace{5pt}\today}
% \date{\vspace{5pt}\formatdate{28}{9}{2017}}

%===================================================================================%
%				THEME COLORS
%===================================================================================%
% specify main colors which are being used by the theme
\definecolor{bordercolor}{HTML}{002699}
\definecolor{fillcolor}{HTML}{002699}

% fix theme black color which affects tikz plots
\definecolor{black}{RGB}{0,0,0}




%===================================================================================%
%				ACTUAL DOCUMENT CONTENT
%===================================================================================%

\begin{document}

\maketitle

%force to add logs to the each frame title
\addtobeamertemplate{frametitle}{}{%
\begin{textblock*}{100mm}(.9\textwidth,-0.9cm)
\includegraphics[height=0.8cm, valign=c]{logos/ccfd3_white.png}\hspace{10pt}\includegraphics[height=0.78cm, valign=c]{logos/symbol-PL-white.pdf}
\end{textblock*}}

\begin{frame}{Frame with itemize}
  \begin{itemize}
  \setlength\itemsep{2em}
      \item uno
      \item dos
      \item tres
      \item quatro
      \item cinco
  \end{itemize}
\end{frame}



\begin{frame}{Equation array}
    \begin{equation*}
        \begin{array}{l}
            y_1 = m_A \pm \sigma \\
            y_2 = m_B \pm \sigma
        \end{array} 
        \Rightarrow
        \begin{array}{l}
        	m_A = y_1 \pm \sigma \\
            m_B = y_2 \pm \sigma
        \end{array}
    \end{equation*}               
\end{frame}

\begin{frame} {Matrices}
    \begin{equation*}
        \mathbf{y}=\mathbf{X} \mathbf{\beta} + \mathbf{\varepsilon}
    \end{equation*}
    
    where:
    \begin{equation*}
        \mathbf{y} = \left[\begin{matrix} y_1 \\ y_2 \end{matrix}\right], 
        \mathbf{X} = \left[\begin{matrix} 1 & 0 \\ 0 & 1 \end{matrix} \right],
        \mathbf{\beta} = \left[\begin{matrix} m_A \\ m_B \end{matrix}\right], 
        \mathbf{\varepsilon} = \left[\begin{matrix} \sigma \\ \sigma \end{matrix}\right]
    \end{equation*}
    
\end{frame}


\begin{frame} {Using "alert" keyword for highlighting word}
   
The matrix \alert{$\mathbf{M}=\mathbf{X}^T\mathbf{X}$} is called \alert{Fisher Information Matrix}.

\end{frame}


\begin{frame}{"empth" for cursive highlight and "alert" for color highlight}
$$ \mathbf{M} = \mathbf{X}^T \cdot \mathbf{X} $$
  \begin{itemize}
      \item \alert{\emph{A-optimum} $$\Psi(M) = tr(M^{-1})$$}
      
      \item \emph{D-optimum}:  $$\Psi(M) =  log(det(M^{-1}))$$
      
      \item \emph{E-optimum}: $$\Psi(M) = \lambda_{max}(M^{-1})$$
      
  \end{itemize}
\end{frame}



\begin{frame}{Example block}
  Some regular text, equation:
  $$\Psi = tr(M^{-1}) = \frac{x_1^2+x_2^2+2} {(x_1-x_2)^2}$$
  
  \hspace{1cm}
  
  \metroset{block=fill}
    
  \begin{exampleblock}{Verification}
  	Some text in the body of the block
  \end{exampleblock}
\end{frame}

\begin{frame}{Tickz algorithm in boxes }
  \begin{center}
		% WARNING: before "\input" command (if tikz code is in the file) required to set up color theme for the below sheme, as bellow:
		% \definecolor{bordercolor}{HTML}{002699}
		% \definecolor{fillcolor}{HTML}{002699}

		\tikzstyle{block} = [rectangle, draw=bordercolor, fill=fillcolor, text=white, minimum width=4em, text centered, rounded corners, minimum height=4em]
		\tikzstyle{oval} = [rectangle, draw, rounded corners, dashed, red]
		\tikzstyle{line} = [draw=bordercolor, ->, ultra thick]
		\tikzstyle{myarrow}=[line width=1.45mm, draw=bordercolor,-triangle 45, postaction={draw=bordercolor, line width=3mm, shorten >=4mm, -}]
  
		\begin{tikzpicture}[scale=1]
      		\node [block] at (0, 4) (sim) {Symulacja};
      		\node [block] at (3, 4) (pod){POD};
      		\node [block] at (6, 4) (basis){Wektory bazowe};
      		\node [block] at (1, 2) (acrit) {A-kryterium};
      		\node [block] at (5, 2) (opti) {Optymalizacja};
      		\node [oval, fit=(acrit) (opti)] {};
      		\node [rectangle, fill=bordercolor, minimum width=3em, minimum height=1em] at (3, 2){};
      		\node [rectangle, fill=bordercolor, minimum width=1em, minimum height=3em] at (3, 2){};
      		\node [block] at (0, 0) (measure) {Pomiar};
      		\node [block, text width=5em] at (3, 0) (lsm) {Metoda najmniejszych kwadratów};
      		\node [block] at (6, 0) (recon) {Rekonstrukcja};
      		\path [myarrow] (sim) -- (pod);
      		\path [myarrow] (pod) -- (basis);
      		\path [myarrow, rounded corners=10mm] (basis.east) -- ++(1,0) |- (opti.east);
      		\path [myarrow, rounded corners=10mm] (acrit.west) -- ++(-2.5,0) |- (measure.west);
      		\path [myarrow] (measure) -- (lsm);
      		\path [myarrow] (lsm) -- (recon);
		\end{tikzpicture}  	
  \end{center}
\end{frame}


\begin{frame}[standout]
    Topic transition slide
\end{frame}


\begin{frame}{Acknowledgment}
\begin{center}
\includegraphics[height=2cm]{logos/ncbir.png} \\
Following research was performed as a part of COOPERNIK project founded by The National Centre for Research and Development.
\end{center}
\end{frame}



\appendix

\begin{frame}{Backup slied}
    Something to cover your back. This slides are not being numbered nor included in progress bar (thin line below title bar)
\end{frame}



\end{document}
